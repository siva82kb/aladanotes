% !TEX root = ../notes_template.tex
\chapter{Linear Transformations}\label{chp:lintrans}

% \minitoc

% Inset a quote for the chapter
\begin{flushright}
\textit{``Though a bit of an exaggeration, it can be said that a mathematical problem can be solved only if it can be reduced to a calculation in linear algebra. And a calculation in linear algebra will reduce ultimately to the solving of a system of linear equations, which in turn comes down to the manipulation of matrices.''}\\
% Draw a small line at the end of the quote
\rule{0.5\textwidth}{.4pt}\\
\textbf{Thomas A Garrity} \small{in \textit{All the Mathematics You Missed.}}
\end{flushright}

I concede that the first two chapters were a bit dry. Hopefully, you will find topics from now on a bit more interest. Linear algebra is the study of linear transformations. We already saw an example of a linear transformation in Section~\ref{sec:ch01-lin-func}. We will look at linear transformations in in their general form and see how matrices can be used to represent and understand them.

\section{What is a linear transformation?}\label{sec:ch03-lin-trans-def}
A \textit{linear transformation} or \textit{linear map} $T$ is a function between between two vector spaces that preserves the operations of vector addition and scalar multiplication. In this course, we will particularly be interested in linear transformation from $\mb{R}^m$ to $\mb{R}^n$, i.e. $T: \mb{R}^m \to \mb{R}^n$. It satisfies the following two properties:
\begin{enumerate}
    \item \textbf{Additivity:} For all $\mf{x}, \mf{y} \in \mb{R}^m$, $T(\mf{x} + \mf{y}) = T(\mf{x}) + T(\mf{y})$.
    \item \textbf{Homogeneity:} For all $\mf{x} \in \mb{R}^m$ and $c \in \mb{R}$, $T(c\mf{x}) = cT(\mf{x})$.
\end{enumerate}
Note that linear transformations allow us to move the transformation operation into the paranthesis for the addition or scalar multiplication of the arguments. The above properties also imply the following:
\begin{itemize}
    \item \textbf{Superposition:} For all $\mf{x}, \mf{y} \in \mb{R}^m$ and $c, d \in \mb{R}$, $T\pp{c\mf{x} + d\mf{y}} = T\pp{c\mf{x}} + T\pp{d\mf{y}} = cT\pp{\mf{x}} + dT\pp{\mf{y}}$.
    \item \textbf{Zero input:} $T(\mf{0}_m) = \mf{0}_n$, where $\mf{0}_m \in \mb{R}^m$ and $\mf{0}_n \in \mb{R}^n$. 
\end{itemize}

The linear functions we considered in Section~\ref{sec:ch01-lin-func} are special cases of linear transformations, where $n = 1$. We will now look at some examples of transformations that are linear and some that are not.

\begin{boxedstuff}
\begin{example}
Consider the transformation $T: \mb{R}^2 \to \mb{R}^2$ defined by $T\pp{\bmx x_1 \\ x_2 \emx} = \bmx 2x_1 \\ 3x_2 \emx$. This transformation is linear because it satisfies the properties of additivity and homogeneity. For example, $T\pp{\alpha \bmx x_1 \\ x_2 \emx + \beta \bmx y_1 \\ y_2 \emx} = T\pp{\bmx \alpha x_1 + \beta y_1 \\ \alpha x_2 + \beta y_2 \emx} = \bmx 2\pp{\alpha x_1 + \beta y_1} \\ 3\pp{\alpha x_2 + \beta y_2} \emx = \alpha \bmx 2x_1 \\ 3x_2 \emx + \beta \bmx 2y_1 \\ 3y_2 \emx = \alpha T\pp{\bmx x_1 \\ x_2 \emx} + \beta T\pp{ \bmx y_1 \\ y_2 \emx}$.
\end{example}

\begin{example}
Consider the transformation $T: \mb{R}^2 \to \mb{R}^2$ defined by $T\pp{\bmx x_1 \\ y \emx} = \bmx x^2 \\ y^2 \emx$. This transformation is not linear because it does not satisfy the property of homogeneity. For example, $T\pp{2\bmx 1 \\ 2 \emx} = T\pp{\bmx 2 \\ 4 \emx} = \bmx 4 \\ 16 \emx \neq 2\bmx 1 \\ 4 \emx = 2T\pp{\bmx 1 \\ 2 \emx}$.
\end{example}
\end{boxedstuff}

Now you can try your hand at the following exercises. Note that to should $T: \mb{R}^m \to \mb{R}^n$ is not a linear transformation, you only need to shown an example that does not satisfied additivity and homogeneity. However, to show that it is linear, you need to demonstrate that $T$ satisfies both properties for all vectors in $\mb{R}^m$. The proof of linearity can be a bit more involved than showing non-linearity.
\begin{boxedstuff}
\begin{problem}
Which of the following transformations are linear?
\begin{enumerate}
    \item $T: \mb{R}^2 \to \mb{R}^2$, $T\pp{\bmx x_1 \\ x_2 \emx} = \bmx x_1 + x_2 \\ x_1 - x_2 \emx$
    \item $T: \mb{R}^2 \to \mb{R}^3$, $T\pp{\bmx x_1 \\ x_2 \emx} = \bmxc x_1 + 1 \\ x_2 + 2 \\ 3x_1 + 2x_2 - 1\emx$
    \item $T: \mb{R}^3 \to \mb{R}^2$, $T\pp{\bmx x_1 \\ x_2 \\ x_3\emx} = \bmx 0 \\ 0 \emx$
\end{enumerate}
\end{problem}

\end{boxedstuff}

\section{Matrices represent linear transformations}\label{sec:ch03-mat-lin-trans}
In Section~\ref{sec:ch01-lin-func}, we say that any linear function $f: \mb{R}^m \to \mb{R}$ can be represented by a fixed $\mf{w} \in \mb{R}^m$, and $f\pp{\mf{x}} = \mf{w}^\top\mf{x}$. We can extend this idea to linear transformations from $\mb{R}^m$ to $\mb{R}^n$. Any linear transformation $T: \mb{R}^m \to \mb{R}^n$ can be represented by a fixed $n \times m$ matrix $\mf{A} \in \mb{R}^{n \times m}$, such that $T\pp{\mf{x}} = \mf{A}\mf{x}$; $\mf{A}$ is the \textit{matrix representation} of the linear transformation $T$.
\begin{equation}
    \begin{split}
        \mf{y} = T\pp{\mf{x}} = \mf{A}\mf{x} &= \bmxc a_{11} & a_{12} & \cdots & a_{1m}\\ a_{21} & a_{22} & \cdots & a_{2m}\\ \vdots & \vdots & \ddots & \vdots\\ a_{n1} & a_{n2} & \cdots & a_{nm} \emx \bmxc x_1 \\ x_2 \\ \vdots \\ x_m\emx \\
        \bmxc y_1 \\ y_2 \\ \vdots \\ y_n \emx &= \bmxc a_{11}x_1 + a_{12}x_2 + \cdots + a_{1m}x_m \\ 
        a_{21}x_1 + a_{22}x_2 + \cdots + a_{2m}x_m \\
        \vdots \\
        a_{n1}x_1 + a_{n2}x_2 + \cdots + a_{nm}x_m\emx 
    \end{split}
    \label{eq:ch03-lin-trans-mat-rep}
\end{equation}
This shows that all linear transformation are essentially a set of simultaneous linear equations.

Let's assume that I give you a Python/Julia function, which takes in vectors from $\mb{R}^m$, does some computation, and returns vectors from $\mb{R}^n$. I also tell to you that this function is linear. You do not want to use this function. If you knew the matrix $\mf{A}$ corresponding to this linear transformation, then you can just do the function computation yourself. But, how can you find the matrix corresponding this function? Remember, in the case of linear functions how we identified the entries of $\mf{w} \in \mb{R}^m$ (Section~\ref{sec:ch01-lin-func})? We had used the unit vectors of $\mb{R}^m$ to get  the $w_i$s. How could we possibly find out the elements of the matrix $\mf{A}$? Turns out, we do the exact same thing, and the elements of $\mf{A}$ will be revealed to us by the output of the linear transformation. If $\mf{a}_i$ is the $i^{th}$ column of $\mf{A}$, then 
$\mf{a}_i = T\pp{\mf{e}_i}, \,\, \mf{e}_i \in \mb{R}^m$. \textcolor{red}{Can you explain why?}

% \begin{boxedstuff}
%     \begin{problem}
%         Consider the linear transformation $T: \mb{R}^m \to \mb{R}^n$,  with the corresponding matrix $\mf{A} = \bmx T\pp{\mf{e}_1} & T\pp{\mf{e}_2} & \cdots & T\pp{\mf{e}_m}\emx \in \mb{R}^{n \times m}$. Unfortunately, all you have is the following set $\lc \rc$ 
%     \end{problem}
% \end{boxedstuff}

\section{Matrix multiplication and linear transformations}\label{sec:ch03-mat-mult-lin-trans}
Consider the following two transformations $T_A$ and $T_B$: 
\[ \begin{split}
    \mf{y} & = \bmx y_1 \\ y_2 \emx = T_A\pp{\mf{x}} = \mf{A}\mf{x} = \bmx a_{11} & a_{12} \\ a_{21} & a_{22}\emx \bmx x_1 \\ x_2 \emx = \bmx a_{11}x_1 + a_{12}x_2 \\ a_{21}x_1 + a_{22}x_2 \emx\\
    \mf{v} &= \bmx v_1 \\ v_2 \emx = T_B\pp{\mf{u}} = \mf{B}\mf{u} = \bmx b_{11} & b_{12} \\ b_{21} & b_{22}\emx \bmx u_1 \\ u_2 \emx = \bmx b_{11}u_1 + b_{12}u_2 \\ b_{21}u_1 + b_{22}u_2 \emx
\end{split}
\]
Let $\mf{v} = T_C = T_A \circ T_B = T_A\pp{T_B\pp{\mf{x}}}$. We can write this as:
\[ \mf{v} = T_C\pp{\mf{x}} = T_A\pp{T_B\pp{\mf{u}}} = T_A\pp{\mf{B}\mf{u}} = \mf{A}\mf{B}\mf{u} \]
This implies that the matrix $\mf{A}\mf{B}$ corresponds to the transformation $T_C$. This is an important result. Matrix multiplication can be seen as the process of composing two linear transformations or applying two linear transoformation one after the other on a vector $\mf{x}$ (from right to left).
\[ \begin{split}
    \mf{v} = \bmx v_1 \\ v_2 \emx &= T_A\pp{T_B\pp{\mf{u}}} = T_A\pp{\bmx b_{11}u_1 + b_{21}u_2 \\ b_{21}u_1 + b_{22}u_2 \emx} = \bmx a_{11}b_{11}u_1 + a_{11}b_{21}u_2 + a_{12}b_{21}u_1 + a_{12}b_{22}u_2 \\ a_{21}b_{11}u_1 + a_{21}b_{21}u_2 + a_{22}b_{21}u_1 + a_{22}b_{22}u_2 \emx \\
    &= \bmx \pp{a_{11}b_{11} + a_{12}b_{21}}u_1 + \pp{a_{11}b_{21} + a_{12}b_{22}}u_2 \\ \pp{a_{21}b_{11} + a_{22}b_{21}}u_1 + \pp{a_{21}b_{21} + a_{22}b_{22}}u_2 \emx = \bmx a_{11}b_{11} + a_{12}b_{21} + a_{11}b_{21} + a_{12}b_{22} \\ a_{21}b_{11} + a_{22}b_{21} + a_{21}b_{21} + a_{22}b_{22} \emx \bmx u_1 \\ u_2 \emx
\end{split} \]
The above equation shows how the strange definition of matrix multiplication comes out beautifully from the composition of linear transformations. Although, we have taken a simple example of $2 \times 2$ matrices, the same idea can be extended to matrices of any size, as long as the dimensions are compatible for multiplication.

\begin{boxedstuff}
    \begin{problem}
        Consider the matrix multiplication, $\mf{C} = \mf{A}\mf{B}$ with $\mf{A} \in \mb{R}^{n \times p}$ and $\mf{B} \in \mb{R}^{p \times m}$. We know that this operation can be seen as the composition of two linear transformations. Using the idea of composition of linear transformations, show that the the $i^{th}$ column of the matrix $\mf{C}$ is given by the linear combination of the columns of $\mf{A}$ with the coefficients from the $i^{th}$ column of $\mf{B}$.
    \end{problem}
\end{boxedstuff}

\section{System of linear equations}\label{sec:ch03-sys-lin-eqn}
An important use of matrices and linear transformations is in solving systems of linear equations. Consider the following system of linear equations which we have seen from our high school years,
\begin{equation}
    \begin{split}
        a_{11}x_1 + a_{12}x_2 + &\cdots + a_{1m}x_m = b_1\\
        a_{21}x_1 + a_{22}x_2 + &\cdots + a_{2m}x_m = b_2\\
        &\vdots\\
        a_{n1}x_1 + a_{n2}x_2 + &\cdots + a_{nm}x_m = b_n
    \end{split}
    \label{eq:ch03-sys-lin-eqn}
\end{equation}
Here, the coefficients $a_{ij}$ and $b_{i}$ are known, and we are interested in solving for the unknowns $x_{j}$. We know how to solve these equations and also know the geometric associated with these equations. Each row represents a hyperplane in $\mb{R}^m$ and the solution, if it exists, is the set of all points in the intersection of the $n$ planes in $\mb{R}^m$.

We can write the above system of linear equations in the matrix form as the following,
\begin{equation}
    \begin{split}
        a_{11}x_1 + \cdots& + a_{1m}x_m = b_1\\
        &\vdots\\
        a_{n1}x_1 + \cdots& + a_{nm}x_m = b_n
    \end{split} \longrightarrow \mf{A}\mf{x} = \mf{b}, \quad \mf{A \in \mb{R}^{n \times m}}, \mf{x} \in \mb{R}^m, \mf{b} \in \mb{R}^n
\end{equation}
The matrix representation offers a new geometrical view of the system of linear equations in Eq~\ref{eq:ch03-sys-lin-eqn}. The above geometrical interpreation of viewing each equation as a hyperplane in $\mb{R}^m$ is referred to as the \textit{row view}. The other geomtric view is the \textit{column view}, which can be easily seen from the following way of writing Eq.~\ref{eq:ch03-sys-lin-eqn}.
\begin{equation}
    \mf{A}\mf{x} = \mf{b} \longrightarrow x_1\mf{a}_1 + x_2\mf{a}_2 + \cdots + x_m \mf{a}_m = \mf{b}
\end{equation}
The vector $\mf{b}$ is a linear combination of the columns of $\mf{A}$, where the $m$ columns $\mf{a}_i$ and $\mf{b}$ are know, and the unknows $x_i$ are the mixture of the linear combination that we are interested in determining.



\subsection{Linear equations in control problems}


\subsection{Linear equations in estimation problems}


\section{Geometry of linear equations}


\section{Solutions of linear equations}


\section{Revisiting linear independence}

\section{Four fundamental subspaces of $\mf{A}$}


\section{Applications}